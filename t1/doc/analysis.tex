\section{Theoretical Analysis}
\label{sec:analysis}

In this section, the circuit shown in Figure \ref{fig:rc} is analysed
theoretically.

\subsection{Mesh Analysis}
Following this method on the 4 meshes, we obtained the equations

\begin{equation}
\begin{cases}
R_1 I_1 + R_3 (I_1+I_2)+R_4(I_1+I_4)=V_a\\
R_6 I_4 + R_7 I_4 -K_c I_4 + R_4 (I_1+I_4)=0\\
I_3=I_d\\
I_2=I_b=K_b V_b = K_b R_3 (I_1+I_2)
\end{cases}
\end{equation}

where $I_1$ to $I_4$ are the currents flowing through each branch as shown in Figure \ref{fig:rc}.

\begin{table}[h]
  \centering
  \begin{tabular}{|l|r|}
    \hline    
    {\bf Name} & {\bf Value [A or V]} \\ \hline
    \input{../mat/tabela1.tex}
 \end{tabular}
 \caption{Operating point. A variable preceded by @ is of type {\em current}
   and expressed in Ampere; other variables are of type {\it voltage} and expressed in
   Volt.}
  \label{tab:op}
\end{table}

\begin{table}[h]
  \centering
  \begin{tabular}{|l|r|}
    \hline    
    {\bf Name} & {\bf Value [A or V]} \\ \hline
    \input{../mat/data.tex}
 \end{tabular}
 \caption{Operating point. A variable preceded by @ is of type {\em current}
   and expressed in Ampere; other variables are of type {\it voltage} and expressed in
   Volt.}
  \label{tab:op}
\end{table}


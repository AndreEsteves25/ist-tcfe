\section{Theoretical Analysis} \label{sec:analysis}

\subsection{Nodal Method at t$<$0}

In general, we considered node analysis (using KCL and Ohm's Law) on the nodes, to discover each voltage associated with each other.
[MAIS CENAS QUE QUEIRAS DIZER ANDRE, NAO PERCEBO UM CU DISTO]

\begin{gather}
\begin{bmatrix}
1 & 0 & 0 & 0 & 0 & 0 & 0 & 0\\
-G_1 & G_1+G_2+G_3 & -G_2& 0 & -G_3 & 0 & 0 & 0\\
0 & -G_2-K_b & G_2 & 0 & K_b & 0 & 0 & 0\\
0 & 0 & 0 & 1 & 0 & 0 & 0 & 0\\
0 & 0 & 0 & 0 & 1 & 0 & K_d G_6 & -1\\
0 & -G_3 & 0 & 0 & G_3+G_4+G_5 & -G_5 & -G_7 & G_7\\
0 & K_b & 0 & 0 & -K_b-G_5 & G_5 & 0 & 0\\
0 & 0 & 0 & 0 & 0 & 0 & G_6+G_7 & -G_7
\end{bmatrix}
\begin{bmatrix}
 V_1\\
 V_2\\
 V_3\\
 V_4\\
 V_5\\
 V_6\\
 V_7\\
 V_8
\end{bmatrix}
=
\begin{bmatrix}
 V_S\\
 0\\
 0\\
 0\\
 0\\
 0\\
 0\\
 0
\end{bmatrix}
\end{gather}

[DESTA VEZ TEMOS AS EQS POR ORDEM?]

Using the scientific programming language GNU Octave to solve the system above, we achieved the following:







\section{Theoretical Analysis} \label{sec:analysis}

\subsection{Nodal Method at t$<$0}

In general, we considered node analysis (using KCL and Ohm's Law) on the nodes, to discover each voltage associated with each other.

[MAIS CENAS QUE QUEIRAS DIZER ANDRE, NAO PERCEBO UM CU DISTO]

\begin{gather}
\begin{bmatrix}
1 & 0 & 0 & 0 & 0 & 0 & 0 & 0\\
-G_1 & G_1+G_2+G_3 & -G_2& 0 & -G_3 & 0 & 0 & 0\\
0 & -G_2-K_b & G_2 & 0 & K_b & 0 & 0 & 0\\
0 & 0 & 0 & 1 & 0 & 0 & 0 & 0\\
0 & 0 & 0 & 0 & 1 & 0 & K_d G_6 & -1\\
0 & -G_3 & 0 & 0 & G_3+G_4+G_5 & -G_5 & -G_7 & G_7\\
0 & K_b & 0 & 0 & -K_b-G_5 & G_5 & 0 & 0\\
0 & 0 & 0 & 0 & 0 & 0 & G_6+G_7 & -G_7
\end{bmatrix}
\begin{bmatrix}
 V_1\\
 V_2\\
 V_3\\
 V_4\\
 V_5\\
 V_6\\
 V_7\\
 V_8
\end{bmatrix}
=
\begin{bmatrix}
 V_S\\
 0\\
 0\\
 0\\
 0\\
 0\\
 0\\
 0
\end{bmatrix}
\end{gather}

[DESTA VEZ TEMOS AS EQS POR ORDEM?]

Using the scientific programming language GNU Octave to solve the system above, we achieved the following:


\begin{table}[!htb]
    %\caption{Global caption}
    \begin{minipage}{.5\linewidth}
      
      \centering
        \begin{tabular}{|c|c|}
        \hline    
        {\bf Node} & {\bf V\textsubscript{i} [V]} \\ \hline
        \input{../mat/tabelaNodes1.tex}
        \end{tabular}
        \caption{Voltage values on each node (t$<$0)}
    \end{minipage}%
    \begin{minipage}{.5\linewidth}
      \centering
        
        \begin{tabular}{|c|c|}
        \hline    
        {\bf Branch} & {\bf I[A]} \\ \hline
        \input{../mat/tabelaIbranches1.tex}
        \end{tabular}
        \caption{Current values on each branch (t$<$0)}
    \end{minipage} 
\end{table}

\subsection{Circuit at t=0}
At t=0, V\textsubscript{S}=0. In order to obtain R\textsubscript{eq}, the equivalent resistance seen by the capacitor, we ran nodal analysis once again, for this time instant, and the computed R\textsubscript{eq} = V\textsubscript{x} / I\textsubscript{x}, where V\textsubscript{x} = V(6)-V(8).

The following tables show the voltages obtained on each node, as well as the data that caracterizes the capacitor.

\begin{table}[!htb]
    %\caption{Global caption}
    \begin{minipage}{.5\linewidth}
      
      \centering
        \begin{tabular}{|c|c|}
        \hline    
        {\bf Node} & {\bf V\textsubscript{i} [V]} \\ \hline
        \input{../mat/tabelaNodes2.tex}
        \end{tabular}
        \caption{Voltage values on each node (t=0)}
    \end{minipage}%
    \begin{minipage}{.5\linewidth}
      \centering
        
        \begin{tabular}{|c|c|}
        \hline    
        {\bf } & {\bf Value} \\ \hline
        \input{../mat/capacitor.tex}
        \end{tabular}
        \caption{Data that caracterizes the capacitor at t=0}
    \end{minipage} 
\end{table}
\subsection{Natural Solution}
It is know that the natural solution for a capacitor in a circuit takes the following form:
\begin{equation}
 V_n(t)=A e^{- \frac{t}{RC}}
\end{equation}

Imposing the boundary condition that at t=0, V\textsubscript{6}(0)= V\textsubscript x, we have:

\begin{equation}
 V_6(t)=V_x e^{- \frac{t}{R_{eq} C}}
\end{equation}

The following plot shows the natural solution within the first 20ms.
\begin{figure}[h] \centering
\includegraphics[width=0.8\linewidth]{t2-3.eps}
\caption{Natural solution t=[0,20]ms}
\label{fig:forced}
\end{figure}

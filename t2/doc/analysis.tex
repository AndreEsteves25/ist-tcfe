\section{Theoretical Analysis}
\label{sec:analysis}

\subsection{Mesh Analysis}

Following this method on the 4 meshes, we obtained the equations (using KVL and Ohm's Law)

\begin{equation}
\begin{cases}
R_1 I_1 + R_3 (I_1+I_2)+R_4(I_1+I_4)=V_a\\
R_6 I_4 + R_7 I_4 -K_c I_4 + R_4 (I_1+I_4)=0\\
I_3=I_d\\
I_2=I_b=K_b V_b = K_b R_3 (I_1+I_2)
\end{cases}
\end{equation}

where $I_1$ to $I_4$ are the currents flowing through each mesh as shown in Figure \ref{fig:rc}. Rearranging these equations in order to the currents, one can obtain the following system:

\begin{gather}
    \begin{bmatrix}
     R_1+R_3+R_4 & R_3 & 0 & R4 \\
     R_4 & 0 & 0 & R_6+R_7-K_c+R_4\\
     0 & 0 & 1 & 0 \\
     K_b R_3 & K_b R_3 -1 & 0 & 0 
    \end{bmatrix}
    \begin{bmatrix}
     I_1\\
     I_2\\
     I_3\\
     I_4
    \end{bmatrix}
    =
    \begin{bmatrix}
     V_a\\
     0\\
     I_d\\
     0
    \end{bmatrix}
\end{gather}

Using the scientific programming language GNU Octave to solve the system above, we achieved the following:

\begin{table}[h]
  \centering
  \begin{tabular}{|l|r|}
    \hline    
    {\bf Mesh Current} & {\bf iI[mA]} \\ \hline
    \input{../mat/tabela1.tex}
 \end{tabular}
 \caption{Currents flowing through each mesh accordingly to Figure \ref{fig:rc}}
  \label{tab:op}
\end{table}

With these results, one can easily obtain the current flowing through each branch.

Note that, once again, the direction of the chosen currents drawn in Figure 1 dictates the signals of the values presented in the next table. In the event of a conflict, namely Vc and R5, between the current flowing in two different meshes, the positive direction was chosen to be upwards and to the left, respectively.

\begin{table}[h]
  \centering
  \begin{tabular}{|l|r|}
    \hline    
    {\bf Branch} & {\bf I[ma]} \\ \hline
    \input{../mat/data.tex}
 \end{tabular}
 \caption{Currents flowing through each branch accordingly to Figure \ref{fig:rc}}
  \label{tab:op}
  \label{tab:op}
\end{table}


\subsection{Node Analysis}
In this Analysis, we enumerated the 8 nodes, from 0 to 7, as it's shown in Figure \ref{fig:rc}. We recognized that the GND is connected to node 0, so $V_{0}=0V$. Consequently $V_1$ is equal to the voltage on voltage source $V_a$. In general, we considered node analysis (using KCL and Ohm's Law) on the nodes, to discover each voltage associated with each other. Another important aspect we considered while analyzing this circuit is that the potential difference between node 4 and node 7 is equal to $V_c$, so we can say that $V_c=V_4-V_7$. We can also apply KCL in the supernode made up by the linearly dependent voltage source ($V_c$), node 4 and node 7, and state that the sum of the currents that go through $R_5$ and $R_4$ with $I_d$ is equal to the sum of the currents passing on $R_3$ and $R_7$.
Putting all this together, we obtained the following equations:

\begin{equation}
\begin{cases}
V_1=V_a\\
\frac{V_1-V_2}{R_1} + \frac{V_3-V_2}{R_2} - \frac{V_b}{R_3}=0\\
K_bV_b - \frac{V_3-V_2}{R_2}=0\\
V_4= V_7+V_c\\

\frac{V_4-V_5}{R_5} - K_bV_b + I_d=0\\

\frac{0-V_6}{R_6} - \frac{V_6-V_7}{R_7} = 0\\
\frac{V_2-V_4}{R_3} + \frac{V_6-V_7}{R_7} - \frac{V_4-0}{R_4}-\frac{V_4-V_5}{R_5}-I_d=0\\
\end{cases}
\end{equation}

Now we substitute $V_b=V_2-V_4$ and $V_c=-K_c\frac{V_6}{R_6}$ and obtained the following equations system, presented in matrix form:

\begin{gather}
    \begin{bmatrix}
     1 & 0 & 0 & 0 & 0 & 0 & 0  \\
     \frac{1}{R_1} & -\frac{1}{R_1}-\frac{1}{R_2}-\frac{1}{R_3} &\frac{1}{R_2} & \frac{1}{R_3} & 0 & 0 & 0\\
     0 & \frac{1}{R_2+K_b} & -\frac{1}{R_2}& -K_b & 0 & 0 & 0\\
     0 & 0 & 0 & 1 & 0 & \frac{K_c}{R_6} & -1\\
     0 & -K_b & 0 & \frac{1}{R_5}+K_b & -\frac{1}{R_5} & 0 & 0\\
     0 & 0 & 0 & 0 & 0 & -\frac{1}{R_6}-\frac{1}{R_7} & \frac{1}{R_7}\\
     0 & \frac{1}{R_3} & 0 & -\frac{1}{R_3}-\frac{1}{R_4}-\frac{1}{R_5} & \frac{1}{R_5} & \frac{1}{R_7} & -\frac{1}{R_7}
    \end{bmatrix}
    \begin{bmatrix}
     V_1\\
     V_2\\
     V_3\\
     V_4\\
     V_5\\
     V_6\\
     V_7
    \end{bmatrix}
    =
    \begin{bmatrix}
     V_a\\
     0\\
     0\\
     0\\
     -I_d\\
     0\\
     I_d
    \end{bmatrix}
\end{gather}


Finally, solving this linear system, using \textit{GNU Octave}, we have the values of each voltage, in the circuit:

\begin{table}[h]
  \centering
  \begin{tabular}{|l|r|}
    \hline    
    {\bf Name} & {\bf Voltage V} \\ \hline
    \input{../mat/tabelaNodes.tex}
 \end{tabular}
 \caption{Table with Voltage Values obtained from resolution of equations system (4)}
  \label{tab:op}
\end{table}

In addition, we can use the given equations, $I_b=K_bV_b$ and $I_c=\frac{V_c}{K_c}$, to confirm values obtained before, in mesh analysis: $I_b=-0.307188$ mA; $I_c=0.937495$ mA.







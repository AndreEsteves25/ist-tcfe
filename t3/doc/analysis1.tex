\section{Theoretical Analysis} \label{sec:analysis}

In this section, the circuit shown in Figure \ref{fig:rc} is analysed theoretically, using Kirchhoff laws, Ohm's Law and the ideal diode model.
Ocateve was also used to obtained the theoretical output values of the envelope detector and the voltage regulator. 

As seen in Figure \ref{fig:rc.pdf}, this circuit can be divided in four parts: \textbf{Voltage Source}, which has an amplitude of 230 V and a frequency of 50 Hz; \textbf{Transformer}, used to transform the supplied voltage into a lower voltage; \textbf{Envelope Detector}, which includes a full wave bridge rectifier and a capacitor ;\textbf{Voltage Regulator}, which consists in a capacitor, a resistor and also a limiter composed by a group of 23 diodes in series.

\subsection{Envelope Detector Analysis}

The envelope detector consist of a full wave bridge rectifier and a capacitor, with $1\mu$F.
The rectifier used is the full wave bridge, composed by four diodes. Ideally, this means that the output(vO) is the absolute value of the input (vS) of this rectifier: vO = |vS|.

The capacitor was used to smooth the wave in order to make it as close to a line as possible.

\begin{table}[h]
  \centering
  \begin{tabular}{|c|c|}
    \hline    
     & { Value } \\ \hline
    \input{../mat/tabela1.tex}
 \end{tabular}
 \caption{Values used for each component and results from envelope detector analysis}
  \label{tab:op}
\end{table}

\begin{figure}[h]
    \centering
    \includegraphics[width = 0.6\linewidth]{envelope.eps}
    \caption{{\it Octave} output: Envelope Dectector Plot}
    \label{fig:OctaveOut}
\end{figure}

\begin{figure}[h]
    \centering
    \includegraphics[width = 0.6\linewidth]{envelope2.eps}
    \caption{{\it Octave} output: Envelope Dectector Plot}
    \label{fig:OctaveOut}
\end{figure}



\subsection{Voltage Regulator Analysis}

The voltage regulator consists of a resistor and a limiter, which itself is composed by 23 diodes connected in series. This component takes advantage of the non-linear diode characteristics to attenuate oscillations in the input signal.

The limiter’s function is to regulate the voltage to the voltage output we want, which is 12V.

\begin{table}[h]
  \centering
  \begin{tabular}{|c|c|}
    \hline    
     & { Value} \\ \hline
    \input{../mat/tabela2.tex}
 \end{tabular}
 \caption{Values used for each component and results from envelope detector analysis}
  \label{tab:op2}
\end{table}


\begin{figure}[h]
    \centering
    \includegraphics[width = 0.6\linewidth]{final.eps}
    \caption{{\it Octave} output: Envelope Dectector Plot}
    \label{fig:OctaveOut}
\end{figure}


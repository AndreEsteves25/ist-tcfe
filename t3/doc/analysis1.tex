\section{Theoretical Analysis} \label{sec:analysis}


Octave was used to obtain the theoretical output values of the envelope detector and the voltage regulator. 



\subsection{Envelope Detector Analysis}

The envelope detector consists of a full-wave bridge rectifier and a capacitor, with $1\mu$F.
The rectifier used is the full-wave bridge, composed by four diodes. Ideally, this means that the output(vO) is the absolute value of the input (vS) of this rectifier: vO = $|vS|$.

The capacitor was used to smooth the wave to make it as close to a line as possible. It was assumed that the capacitor discharges through a series of resistors (composed by the resistor $Rb1$ in series with 25 diodes).

\begin{figure}[h] \centering
  \begin{minipage}{.5\textwidth}
    \includegraphics[width=.9\textwidth]{envelope.eps}
    \caption{Envelope voltage ripple}
    \label{fig:simenv}
    \end{minipage}%
  \begin{minipage}{.5\textwidth}
  \centering
    \includegraphics[width=.9\textwidth]{envelope2.eps}
    \caption{Bridge and envelope output}
    \label{fig:compenv}
      \end{minipage}%
\end{figure}




\subsection{Voltage Regulator Analysis}

The voltage regulator consists of a resistor and a limiter (25 diodes in series). This subcircuit takes advantage of the non-linear diode characteristics to attenuate oscillations in the input signal.

We followed the incremental diode model to analyse this subcircuit, assuming that the mean output voltage is 12V (confirmed by the ngspice simulation).


\begin{figure}[h] \centering
  \begin{minipage}{.5\textwidth}
    \includegraphics[width=.9\textwidth]{final.eps}
    \caption{Final voltage ripple}
    \label{fig:simenv}
    \end{minipage}%
  \begin{minipage}{.5\textwidth}
  \centering
    \includegraphics[width=.9\textwidth]{final2.eps}
    \caption{Final output (DC level and its deviation)}
    \label{fig:compenv}
      \end{minipage}%
\end{figure}



\begin{table}[h]
  \centering
  \begin{tabular}{|c|c|}
    \hline    
     & { Value} \\ \hline
    \input{../mat/tabela.tex}
 \end{tabular}
 \caption{Results of theoretical analysis}
  \label{tab:op2}
\end{table}

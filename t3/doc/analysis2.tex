\subsection{Frequecy Analysis}

At last, it was studied the frequency responses.

To obtain the phasor in each node, it was computed a loop (with 100 iterations, one for each frequency) with a nodal analysis similar to the previous one.

\subsubsection{Magnitude Response}

We computed the absolute value of each relevant phasor to obtain the magnitude response.

\begin{figure}[h] \centering
\includegraphics[width=0.5\linewidth]{t2-6magn.eps}
\caption{Magnitude response}\label{fig:magn}
\end{figure}

Here we can see that the capacitor's terminal voltage decreases drastically as f rises. This radical change isn't felt this strongly in node 6.


\subsubsection{Phase Response}

Finally, we computed the argument of each relevant phasor to obtain the phase response.
\begin{figure}[h] \centering
\includegraphics[width=0.5\linewidth]{t2-6phas.eps}
\caption{Magnitude response}\label{fig:phase}
\end{figure}

As we can see, the capacitor's terminal voltage goes to -90 degrees as f rises, and the voltage at node 6 is asymptotic to -180 degrees. When the frequency is between 100Hz and 1000Hz, the phases variation increase



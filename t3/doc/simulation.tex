\section{Simulation Analysis}
\label{sec:simulation}

The circuit under analysis was also simulated using Ngspice (revision 31) and maintaining the already mentioned node convention (figure REF-RC). The forward subsections establish a comparison between theoretical and simulation results.

\subsection{Envelope Detector}




\begin{figure}[h] \centering
  \begin{minipage}{.5\textwidth}
    \includegraphics[width=.9\textwidth]{../t3env.eps}
    \caption{Magnitude response in dB (related to the frequency of Vs) - simulation}
    \label{fig:simenv}
    \end{minipage}%
  \begin{minipage}{.5\textwidth}
  \centering
    \includegraphics[width=.99\textwidth]{../t3env.eps}
    \caption{Magnitude responsein dB (related to the frequency of Vs) - theoretical analysis }
    \label{fig:compenv}
      \end{minipage}%
\end{figure}

\subsection{Voltage Regulator}

Having simulated the circuit for t$<$0, there must be performed an intermediate step at t=0, before jumping to the transient analysis imposed by the sinusoidal output of Vs at t$\geq$0. This additional step is the calculation of the initial conditions for the capacitor' terminals at the transient phase. These values are obtained having Vs=0, and replacing the capacitor by a voltage source that imposes the voltage difference verified between the capacitor' terminals at t$<$0 (nodes 6 and 8). Given the conditions of simplicity, this step, much like the previous one, shows a great deal of consistency between simulation (table  and theoretical (table ref{comp2}) results. 




\begin{figure}[h] \centering
  \begin{minipage}{.5\textwidth}
    \includegraphics[width=.9\textwidth]{../t3out.eps}
    \caption{Magnitude response in dB (related to the frequency of Vs) - simulation}
    \label{fig:simout}
    \end{minipage}%
  \begin{minipage}{.5\textwidth}
  \centering
    \includegraphics[width=.99\textwidth]{../t3out.eps}
    \caption{Magnitude responsein dB (related to the frequency of Vs) - theoretical analysis }
    \label{fig:compout}
      \end{minipage}%
\end{figure}

\begin{figure}[h] \centering
  \begin{minipage}{.5\textwidth}
    \includegraphics[width=.9\textwidth]{../t3outcomp.eps}
    \caption{Magnitude response in dB (related to the frequency of Vs) - simulation}
    \label{fig:simdev}
    \end{minipage}%
  \begin{minipage}{.5\textwidth}
  \centering
    \includegraphics[width=.99\textwidth]{../t3outcomp.eps}
    \caption{Magnitude responsein dB (related to the frequency of Vs) - theoretical analysis }
    \label{fig:compdev}
      \end{minipage}%
\end{figure}


\begin{table}[!htb]
  \begin{minipage}{.5\linewidth}
     \centering
  \begin{tabular}{|c|c|}
    \hline    
    {\bf Parameter} & {\bf Value} \\ \hline
    \input{../sim/sim_tab.tex}
 \end{tabular}
 \caption{Simulation results. A variable preceded by @ is of type {\em current}
   and expressed in Ampere; other variables are of type {\it voltage} and expressed in
   Volt.}
 \label{tab:sim}
  \end{minipage}%
  \hspace{3mm}
    \begin{minipage}{.5\linewidth}
      \centering
        \begin{tabular}{|c|c|}
    \hline    
    {\bf Parameter} & {\bf Value} \\ \hline
    \input{../sim/sim_tab.tex}
 \end{tabular}
        \caption{Current values on each branch and voltage values on each node(t$<$0)}
        \label{comppar}
    \end{minipage} 
\end{table}



\newpage


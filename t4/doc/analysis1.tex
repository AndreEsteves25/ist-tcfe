\section{Theoretical Analysis} \label{sec:analysis}
 


\subsection{Gain Stage}
\subsubsection{Operating Point}
The bias circuit V\textsubscript{cc}, R\textsubscript1, R\textsubscript2 will determine the base voltage V\textsubscript B and ensure the BEJ is on.

To be easier to analyse the bias circuit we can ignore the capacitors and make a Thévenin equivalent, replacing resistors R\textsubscript{1} and R\textsubscript{2}, which are in parallel, with an equivalent resistor R\textsubscript{B}:
\begin{equation}
R_B = \frac{R_{1} R_{2}}{R_{1}+R_{2}}
\end{equation}

Looking at the circuit as a voltage divider, we have
\begin{equation}
V_{eq} = \frac{R_{2}}{R_{1}+R_{2}} V_{cc}
\end{equation}

To calculate the current passing through the node \textit{e} we know that I\textsubscript E =(1+$\beta_F$)I\textsubscript B.

Using the mesh analysis for the mesh on the left side and assuming that the current is going clockwise,
\begin{equation}
V_{eq} + R_B I_{B1} + V_{BEON} + R_{E1} I_{E1} = 0 \Leftrightarrow I_{B1} = \frac{V_{eq}-V_{BEON}}{R_B + (1+\beta_{FN}) R_{E1}}
\end{equation}

From this transistor model, we also know that
\begin{equation}
 I_{C1}=\beta_{FN}*I_{B1};
\end{equation}

\begin{equation}
 I_{E1}=(1+\beta_{FN})*I_{B1};
\end{equation}

\begin{equation}
 V_{c}=V_{cc}-R_{c}I_{C1};
\end{equation}

In the operating point calculations, we are also checking if the transistor is in the forward active region:  V\textsubscript{ce}=\input{../mat/FAR.tex} 
\subsubsection{Incremental Analysis}
\subsection{Output Stage}

\subsection{Frequency Response}





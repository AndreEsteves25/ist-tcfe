\section{Theoretical Analysis} \label{sec:analysis}
 


\subsection{Gain Stage}
\subsubsection{Operating Point}
The bias circuit V\textsubscript{cc}, R\textsubscript1, R\textsubscript2 will determine the base voltage V\textsubscript B and ensure the BEJ is on.

To be easier to analyse the bias circuit we can ignore the capacitors and make a Thévenin equivalent, replacing resistors R\textsubscript{1} and R\textsubscript{2}, which are in parallel, with an equivalent resistor R\textsubscript{B}:
\begin{equation}
R_B = \frac{R_{B1} R_{B2}}{R_{B1}+R_{B2}}
\end{equation}

Looking at the circuit as a voltage divider, we have
\begin{equation}
V_{eq} = \frac{R_{B2}}{R_{B1}+R_{B2}} V_{cc}
\end{equation}

To calculate the current passing through the node \textit{e} we know that I\textsubscript E =(1+$\beta_F$)I\textsubscript B.

Using the mesh analysis for the mesh on the left side and assuming that the current is going clockwise,
\begin{equation}
V_{eq} + R_B I_{B1} + V_{BEON} + R_{E1} I_{E1} = 0 \Leftrightarrow I_{B1} = \frac{V_{eq}-V_{BEON}}{R_B + (1+\beta_{FN}) R_{E1}}
\end{equation}

From this transistor model, we also know that
\begin{equation}
 I_{C1}=\beta_{FN}*I_{B1};
\end{equation}

\begin{equation}
 I_{E1}=(1+\beta_{FN})*I_{B1};
\end{equation}

\begin{equation}
 V_{c}=V_{cc}-R_{c}I_{C1};
\end{equation}

In the operating point calculations, we are also checking if the transistor is in the forward active region:  V\textsubscript{ce} = \input{../mat/FAR.tex} $>$ 0.7= V\textsubscript{BEon}, as we inteded.
\subsubsection{Incremental Analysis}
Following the operating point analysis, we can compute, the following incremental data, which concerns the bipolar transitor small signal model (AC)
\begin{align*} 
g_{m1}= I_{C1}/V_{T}\\
r_{\pi 1}=\beta_{FN}/g_{m1}\\
r_{o 1}=V_{AFN}/I_{C1}
\end{align*}

where $g_{m1}$ is the transconductance, $V_{T}$=25mV is the termal voltage, $r_{\pi 1}$ is the input incremental impedance, $\beta_{FN}$=178.7 is the common emmiter current gain, $r_{o 1}$ is the output incremental resistance, and finally, $V_{AFN}$=69.7V is the forward mode early voltage.


Assuming that the capacitor C\textsubscript{E} is a short-circuit when we are analysing the AC component, we have R\textsubscript{E}=0,

\begin{align*} 
\frac{vo}{vi}=-g_{m1} (R_C\parallel r_o) \frac{r_\pi \parallel R_B} {R_s+r_\pi\parallel R_B} v_s
\end{align*}

\begin{table}[!htb]
\centering
  \begin{tabular}{|c|c|}
    \hline    
    {\bf Parameter} & {\bf Value} \\ \hline
    \input{../mat/gainstage.tex}
 \end{tabular}
 \caption{Results of simulation analysis}\label{tab:gainstage}
\end{table}

\subsection{Output Stage}

\subsection{Frequency Response}





\section{Conclusion}
\label{sec:conclusion}
In this laboratory assignment, the objective of build and analyse the working principle of an audio amplifier was successfully achieved.
The cost of the circuit was \input{../mat/custo.tex} and the merit 909.855.
Comparing the results from both the theoretical analysis using Octave and the circuit simulation using ngspice, we can see in the figures 5 and 6 and tables 1,2,3 and 4, that they appear to slightly differ.

In both analysis the output voltage was close to $12 V$ and had a small ripple. The shape of the graphs obtained in both of them was very similar and the frequencies and the phases appear to be the same.

However, the theoretical model gives a lower value for both the input and output impendance and a smaller bandwidth, but a lot bigger value for the gain. 

We can conclude that there are some differences between the theoretical and simulation analysis. These variations are due to the various aproximations made in the theoretical analysis, in the ideal transistor model for example, while the simulation assume that the circuit is linear and uses a comparatively accurate spice model.  


%merry chrislter
%$https://www.youtube.com/watch?v=_Z-Nu351j58$

%\lipsum[1-1]

\section{Conclusion}
\label{sec:conclusion}
In this laboratory assignment, the objective of building and analyze the working principle of an audio amplifier was successfully achieved.
The cost of the circuit was \input{../mat/custo.tex} and the merit 909.855, so it was also achieved a good relation between the price of the circuit and the quality of the output signal , so it doesn't have the visible distortion of the input sines wave.
Comparing the results from both the theoretical analysis using Octave and the circuit simulation using ngspice, we can see in figures 5 and 6 and tables 1,2,3 and 4, that they appear to slightly differ.

In both analyses, the output voltage was close to $12 V$ and had a small ripple. The shape of the graphs obtained in both of them was very similar and the frequencies and the phases appear to be the same.

However, the theoretical model gives a lower value for both the input and output impedance, comparing to the simulation one, but both of these output values are compatible with a 8 $\Omega$ load. It also has a smaller bandwidth, but a lot bigger value for the gain, due to the less sensitive model used on Octave. 

We can conclude that there are some differences between the theoretical and simulation analysis. These variations are due to the various approximations made in the theoretical analysis, for example: it was considered that the capacitors behaved as open-circuits and short-circuits, so we can have cutoff frequencies f\textsubscript{Low}=0 and f\textsubscript{High}= $\infty$, respectevily. On the other hand, the simulation assumes that the circuit is linear and uses a comparatively accurate and complex models.  




%merry chrislter
%$https://www.youtube.com/watch?v=_Z-Nu351j58$

%\lipsum[1-1]

\section{Simulation Analysis}
\label{sec:simulation}

The circuit under analysis was also simulated using Ngspice (revision 31) and maintaining the already mentioned node convention (figure \ref{fig:rc}). The forward subsections establish a comparison between theoretical and simulation results.

\subsection{Gain Stage}
From the simulation, one can conclude that a great deal of small-signal amplification occurs at the gain stage which is exactly this section's function. However, the signal has a huge DC deviation which is undesirable. Also, according to the rough estimates of the theoretical analysis, this stage has a huge output impedance which means that it enclosures the voltage out of the real output zone. This is why we need the output stage to impose enough signal amplitude arriving at the load. A graph of the signal at node C (the signal at the end of the gain stage) in a 100Hz regime is shown below. Bellow that is shown the frequency response of the gain stage's output which has some clear irregularities but a large bandwidth. These irregularities will also be corrected at the output stage.

\begin{figure}[h] \centering
\includegraphics[width=0.9\linewidth]{vcoll.eps}
\vspace{-5mm}
\caption{Gain Stage output at 100Hz}
\label{fig:vcoll}
\end{figure}

\begin{figure}[h] \centering
\includegraphics[width=0.9\linewidth]{vdbcoll.eps}
\vspace{-5mm}
\caption{Gain Stage output-frequency response}
\label{fig:vdbcoll}
\end{figure}

\newpage
\subsection{Output Stage}

The simulation shows that the output stage accomplishes its essential function. The output impedance is reduced (table \ref{tab:z}). And as shown in figure \ref{fig:vout} the output signal has a gain of roughly 50 (the input signal has an amplitude of 10 volts) and has no DC deviation and negligible distortion. The output signal also shows a large and plane bandwidth in the frequency response as desired (figure \ref{fig:vdbout}). The comparison between theoretical and simulation analysis will be discussed in detail in the conclusion section. It is also noted that the input impedance is high compared to the 100 Ohms of the input resistance which is also a desired feature.

\begin{figure}[h] \centering
\includegraphics[width=0.9\linewidth]{vout.eps}
\vspace{-5mm}
\caption{Output at 100Hz}
\label{fig:vout}
\end{figure}

\begin{figure}[h] \centering
  \begin{minipage}{.45\textwidth}
    \includegraphics[width=.8\textwidth]{vdbout.eps}
    \caption{Output frequency response - theoretical analysis}
    \label{fig:vdbout}
  \end{minipage}%
    \hspace{2 mm}
  \begin{minipage}{.45\textwidth}
  \centering
    \includegraphics[width=.9\textwidth]{frequencyresponse.eps}
    \caption{Output frequency response - theoretical analysis}
    \label{fig:compvdbout}
      \end{minipage}%
\end{figure}

\begin{table}[!htb]
  \begin{minipage}{.5\linewidth}
     \centering
  \begin{tabular}{|c|c|}
    \hline    
    {\bf Parameter} & {\bf Value} \\ \hline
    \input{../sim/zi_tab.tex}
    \input{../sim/zo_tab.tex}
 \end{tabular}
 \caption{Impedance (in Ohms) -  simulation analysis}
 \label{tab:z}
  \end{minipage}
    \hspace{2 mm}
    \begin{minipage}{.5\linewidth}
      \centering
        \begin{tabular}{|c|c|}
    \hline    
    {\bf Parameter} & {\bf Value} \\ \hline
     \input{../mat/total.tex}
 \end{tabular}
        \caption{Impedance (in Ohms) - theoretical analysis}
        \label{tab:compz}
    \end{minipage} 
\end{table}


\newpage
\subsection{Performance and Quality}
Besides the already presented input and output impedances, the performance and quality of this audio amplifier can be measured by cost, gain, lower-cutoff frequency, bandwidth, and merit (cost-benefit relation). In table \ref{tab:merit} these parameters are shown (and compared with the theoretical calculations). The gain is quite good (about 50 times) and the band of amplified frequencies contains all frequencies that a human can hear which is the maximum desirable band. At last, the merit is quite high. But the following should be stated: the figure of merit has some limitations. For example, if we reduced Rc by a factor of ten the figure of merit would become a great deal greater (more than its double), however, we would have an overall worse amplifier because it would have less gain. The merit would rise due to a great rise in the bandwidth but this expansion would to a region of frequencies that no human can hear, and who wants an amplifier with less gain just to have a uselessly large bandwidth?


\begin{table}[!htb]
  \begin{minipage}{.5\linewidth}
     \centering
  \begin{tabular}{|c|c|}
    \hline    
    {\bf Parameter} & {\bf Value} \\ \hline
    \input{../sim/merit_tab.tex}
 \end{tabular}
 \caption{Results of simulation analysis}
 \label{tab:merit}
  \end{minipage}%
    \hspace{2 mm}
    \begin{minipage}{.5\linewidth}
      \centering
        \begin{tabular}{|c|c|}
    \hline    
    {\bf Parameter} & {\bf Value} \\ \hline
    \input{../mat/merit.tex}
 \end{tabular}
        \caption{Results of theoretical analysis}
        \label{compmerit}
    \end{minipage} 
\end{table}

\newpage

\section{Simulation Analysis}
\label{sec:simulation}

The circuit under analysis was also simulated using Ngspice (revision 31) and maintaining the already mentioned node convention (figure \ref{fig:rc}). The forward subsections establish a comparison between theoretical and simulation results.
First of all, it was computed the output of the circuit for an input sinusoidal signal with 10 mV of amplitude and 1000 Hz of frequency. The result may be examined in figure \ref{fig:vout}. As it looks, it has little distortion. The gain will be examined next, in the frequency response study.



\begin{figure}[h] \centering
\includegraphics[width=0.4\linewidth]{vout.eps}%0.9
%\vspace{-5mm}
\caption{Output voltage at 100Hz}
\label{fig:vout}
\end{figure}
%\vspace{-2mm}

%\begin{figure}[h] \centering
%\includegraphics[width=0.9\linewidth]{vdbcoll.eps}
%\vspace{-5mm}
%\caption{Gain Stage output-frequency response}
%\label{fig:vdbcoll}
%\end{figure}

As it may be seen in figure \ref{fig:vdbout} the circuit accomplishes with few deviations what it proposes itself to do. It filters high and low frequencies and amplifies with a gain close to 40 dB its band' central frequency wich is close to the desired 1000 Hz (table \ref{tab:sim}) and as expected the central frequency as negligible offset (figure \ref{fig:vpout}). It is also notable that the theoeretical analysis has a close estimative to the simulation  results (figures \ref{fig:compvdbout}, \ref{fig:compvpout} and table \ref{tab:compsim}).

\begin{figure}[h] \centering
  \begin{minipage}{.45\textwidth}
    \includegraphics[width=.8\textwidth]{vdbout.eps}
    \caption{Output voltage - frequency response (amplitude) (simulation result)}
    \label{fig:vdbout}
  \end{minipage}%
    \hspace{2 mm}
  \begin{minipage}{.45\textwidth}
  \centering
    \includegraphics[width=1.0\textwidth]{T.eps}
    \caption{Output voltage - frequency response (amplitude) (theoretical result)}
    \label{fig:compvdbout}
      \end{minipage}%
\end{figure}

\begin{figure}[h] \centering
  \begin{minipage}{.45\textwidth}
    \includegraphics[width=.8\textwidth]{vpout.eps}
    \caption{Output voltage -frequency response (phase) (simulation result)}
    \label{fig:vpout}
  \end{minipage}%
    \hspace{2 mm}
  \begin{minipage}{.45\textwidth}
  \centering
    \includegraphics[width=1.0\textwidth]{Tphase.eps}
    \caption{Output voltage -frequency response (phase) (theoretical result)}
    \label{fig:compvpout}
      \end{minipage}%
\end{figure}

\newpage

\begin{table}[!htb]
  \begin{minipage}{.5\linewidth}
     \centering
  \begin{tabular}{|c|c|}
    \hline    
    {\bf Parameter} & {\bf Value} \\ \hline
    \input{../sim/sim_tab.tex}
    \input{../sim/sim2_tab.tex}
 \end{tabular}
 \caption{Essential results (simulation analysis)}
 \label{tab:sim}
  \end{minipage}
    \hspace{2 mm}
    \begin{minipage}{.5\linewidth}
      \centering
        \begin{tabular}{|c|c|}
    \hline    
    {\bf Parameter} & {\bf Value} \\ \hline
    \input{../mat/final.tex}
 \end{tabular}
        \caption{Essential results (theoretical analysis)}
        \label{tab:compsim}
    \end{minipage} 
\end{table}


